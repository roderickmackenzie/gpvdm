\section{Changing the model}
\subsection{Editing the source code}
You can download the source code from my git hub repository \url{https://github.com/roderickmackenzie/gpvdm}.  If you do add new features yourself, please do send me patches and I will do my best to include your improvements in the main source tree.  I plan to make the source more user friendly by adding doxgen type comments, but as of yet source code comments are few and far between simply because of lack of time on my part to document.  If you have questions send me questions and I will do my best to answer.

\subsection{The structure of the model}
The model is divided into two parts, the graphical interface (GUI) and the back end solver (see \ref{fig:structureofthemodel}).  The role of the GUI, is simply to edit the input files (sim.gpvdm) and view the results.  It is the back end solver which does all the computation.  The file C:$\backslash$gpvdm$\backslash$gpvdm.exe, contains the GUI, and is a python program using a QT widget set, compiled into a windows executable.  The file C:$\backslash$gpvdm$\backslash$gpvdm\_core.exe is the back end solver which is written in C.
\begin{figure}
\centering
\includegraphics[width=\textwidth]{./images/architecture.png}
\caption{The structure of the model}
\label{fig:structureofthemodel}
\end{figure}
If you want to run the model from the command line, make a new simulation using the GUI, then close the GUI, and open up the terminal window.  Navigate to the directory where you saved your simulation, then enter C:$\backslash$gpvdm$\backslash$gpvdm\_core.exe in the command line.  gpvdm should then run in the terminal.

\subsection{Structure of the optical model}
I have split the optical model up into different dynamically loadable modules to so that you can write your own optical modules without too much work.  In linux these are .so files and in windows they are .dll files, these are kept in the 'light' directory.  I've not documented the interface of the plugins but if you start looking at light\_interface.c it should be pretty clear.

\subsection{Structure of the electrical solver}
I've broken the electrical model up into various plugins again to make it easier to write extra modules.  They are dynamically loaded as they are needed.  The Newton solvers and matrix inverting libraries are also plugins so they can be swapped out.


\subsection{Running gpvdm on a cluster}
Gpvdm will run on a cluster, this can be handy when doing large numbers of simulations.  To do this you need to download the cluster management code from  \url{https://github.com/roderickmackenzie/simpleclustercode}.  The clustering code as it stands is pretty much undocumented, but it should be possible to get it going.  You will have to recompile gpvdm to run on the cluster though.


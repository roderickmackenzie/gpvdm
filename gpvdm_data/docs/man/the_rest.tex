
\section{Text past this point needs to be rewritten}





\newpage



\subsubsection{1D, 2D and 3D simulations with gpvdm}

When deciding if you should perform 1D, 2D or 3D, simulations, consider the dimensionality of your problem.  For example if you consider a solar cell, it is only a few micros thick, and there is rapid variation in the structure, charge densities, mobilities, and doping as a function of depth (y).  However, the structure will not vary very quickly in the lateral (xz) plane.  Therefore, in general  to capture all interesting effects present within a solar cell one only needs a 1D model.  If one now considers OFETs, there is both vertical an lateral current flow, therefore one can not get away with a 1D model any more, as one must simulate both vertical current flow, and current between the source and the drain, thus one needs a 2D simulation.  As the number of dimensions increases, computation speed will decrease, therefore my general advice is to use the minimum number of dimensions possible to solve your problem.


\section{The user interface}
\subsubsection{Is Langevin recombination a good way of describing recombination OPV devices?}
In my view Langevin recombination is in general a really bad way to describe recombination in OPV devices.  This is because the mechanism assumes Brownian motion of electrons and holes and that charge carriers of opposite polarity will recombine when they get close enough to fall into each others electrostatic field.  This picture assumes the charge carriers are free and completely neglects the influence of trap states.  I therefore think Langeving recombination should be avoided in OPVs.
But in dx.doi.org/10.1021/jp200234m you used Langevin recombination - why?: In this paper I allowed the mobility in the Langevin expression to vary as a function of carrier density i.e.
\begin{equation}
R_{free}=q k_{r}\frac{(\alpha \mu_e(n)+\beta \mu_h(n)) n_{tot} p_{tot}}{2\epsilon_0\epsilon_r}
\end{equation}

I then by defining a mobility edge and assuming any carrier below the mobility edge could not move and any carrier above it could.  I could define the averaged electron/hole mobility as: 

\begin{equation}
\mu_e(n)=\frac{\mu_e^0 n_{free}}{n_{free}+n_{trap}}
\end{equation}

and

\begin{equation}
\mu_h(n)=\frac{\mu_h^0 p_{free}}{p_{free}+p_{trap}}
\end{equation}

and if one assumes the density of free charge carriers is much smaller than the density of trapped charge carriers one can arrive at

\begin{equation}
R(n,p)=q k_{r}\frac{(\alpha \mu_e^0 n_{free} p_{trap}+\beta \mu_h p_{free} n_{trap}) }{2\epsilon_0\epsilon_r}
\end{equation}

Thus by making the mobility carrier density dependent we arrive at an expression for Langeving recombination that's dependent upon the density of free and trapped carriers (i.e. $n_{free} p_{trap}$ and $ p_{free} n_{trap}$) This is in principle the same as SRH recombination (i.e. a process involving free electrons (holes) recombining with trapped holes (electrons)).  This was a nice simple approach and it worked quite well in the steady state.  However, to make this all work I had to assume all electrons (holes) at any given position in space had a single quasi-Fermi level, which meant they were all in equilibrium with each other.  For this to be true, all electrons (holes) would have to be able to exchange energy with all other electrons (holes) at that position in space and have an infinite charge carrier thermalization velocity.  This seemed like an OK assumption in steady state when electrons (holes) had time to exchange energy, however once we start thinking about things happening in time domain, it becomes harder to justify because there are so many trap states in the device it is unlikely that charge carriers will be able to act as one equilibrated gas with one quasi-Fermi level.  On the other hand the SRH mechanism does not make this assumption, so it is probably a better description of recombination/trapping.  I would also add that I have never found a situation in OPV device modeling where SRH recombination was unable to describe the device in question.  Conclusion: SRH is better than Langevin.  


\subsubsection{Should I trust the results of gpvdm?}
Yes!  The model it's self has been verified against experiment [there are over 20 publications doing this, in steady state, time domain (us-fs time scales), and fx-domain]. The basic drift-diffusion solver was cross checked and compared against other drift diffusion models, and the accuracy compared down to 6-9 dp.  While the optical model has been compared to analytical solutions of Maxwell's equations.  The SRH model has also been compared against analytical models.  If the answers you are getting out of gpvdm are odd, then I would suggest to take a look at the input parameters.  If your efficienceis are high, try increasing the number of trap states, the recombination cross sections or reducing the e/h mobilites.  Finally, I would also recommend always running the latest version, and keeping an eye on the twitter stream for bug announcements.



\subsection{Can I use the model to simulate my exotic* material system/contacts?}
The short answer is yes.  The model is an effective medium model, meaning that it does not simulate the details of the medium, rather it approximates the medium with a set of electrical parameters.  For example, when simulating an organic solar cell, it does not simulate every detail of the BHJ, rather it just assumes an effective mobility, density of states, recombination cross sections, trapping cross sections and so on...  So if you can find electrical parameters to aproximate your material system (or guess them), there is nothing stopping you using gpvdm to simulate any exotic device/material.  The same goes for the contacts, the model simulates the contacts simply as a charge density. So if you have fancy graphene contacts which inject lots of charge, use a high majority carrier density on the contacts.  Where as if you have some dirty old ITO contacts may be drop the majority carrier density a bit.

\newpage



\subsubsection{1D position space output}
\paragraph{Band structure}
\textbf{Ec.dat}:LUMO-position\newline
x-axis:Position($nm$)\newline
y-axis:Electron Energy($eV$)\newline
\newline
\textbf{Efield.dat}:Material number - position\newline
x-axis:Position($nm$)\newline
y-axis:Number($au$)\newline
\newline
\textbf{Eg.dat}:Band gap-position\newline
x-axis:Position($nm$)\newline
y-axis:Electron Energy($eV$)\newline
\newline
\textbf{Ev.dat}:HOMO-position\newline
x-axis:Position($nm$)\newline
y-axis:Electron Energy($eV$)\newline
\newline
\textbf{Fi.dat}:Equlibrium Fermi-level - position\newline
x-axis:Position($nm$)\newline
y-axis:Energy($eV$)\newline
\newline
\textbf{Fn.dat}:Electron quasi Fermi-level position\newline
x-axis:Position($nm$)\newline
y-axis:Electron Energy($eV$)\newline
\newline
\textbf{Fp.dat}:Hole quasi Fermi-level position\newline
x-axis:Position($nm$)\newline
y-axis:Electron Energy($eV$)\newline
\newline
\textbf{phi.dat}:Potential\newline
x-axis:Position($nm$)\newline
y-axis:Potential($V$)\newline
\newline
\paragraph{Chaerge density}
\textbf{dn.dat}:Change in free electron population - position\newline
x-axis:Position($nm$)\newline
y-axis:Carrier density($m^{-3}$)\newline
\newline
\paragraph{Charge density}
\textbf{Nad.dat}:Doping - position\newline
x-axis:Position($nm$)\newline
y-axis:Doping density($m^{-3}$)\newline
\newline
\textbf{dnt.dat}:Excess electron density - position\newline
x-axis:Position($nm$)\newline
y-axis:Electron density($m^{-3}$)\newline
\newline
\textbf{dp.dat}:Change in free hole population - position\newline
x-axis:Position($nm$)\newline
y-axis:Carrier density($m^{-3}$)\newline
\newline
\textbf{dpt.dat}:Excess electron density - position\newline
x-axis:Position($nm$)\newline
y-axis:Hole density($m^{-3}$)\newline
\newline
\textbf{n.dat}:Total hole density - position\newline
x-axis:Position($nm$)\newline
y-axis:Carrier density($m^{-3}$)\newline
\newline
\textbf{nt.dat}:Trapped electron carrier density - position\newline
x-axis:Position($nm$)\newline
y-axis:Carrier density($m^{-3}$)\newline
\newline
\textbf{p.dat}:Total hole density - position\newline
x-axis:Position($nm$)\newline
y-axis:Carrier density($m^{-3}$)\newline
\newline
\textbf{pt.dat}:Trapped hole carrier density - position\newline
x-axis:Position($nm$)\newline
y-axis:Carrier density($m^{-3}$)\newline
\newline
\paragraph{Material parameters}
\textbf{epsilon\_r.dat}:Relative permittivity - position\newline
x-axis:Position($nm$)\newline
y-axis:Relative permittivity($au$)\newline
\newline
\textbf{mu\_n.dat}:Electron mobility - position\newline
x-axis:Position($nm$)\newline
y-axis:Electron mobility($m^{2} V^{-1} s^{-1}$)\newline
\newline
\textbf{mu\_n\_ft.dat}:Electron mobility free/all- position\newline
x-axis:Position($nm$)\newline
y-axis:Mobility($m^{2} V^{-1} s^{-1}$)\newline
\newline
\textbf{mu\_p.dat}:Hole mobility - position\newline
x-axis:Position($nm$)\newline
y-axis:Hole mobility($m^{2} V^{-1} s^{-1}$)\newline
\newline
\textbf{mu\_p\_ft.dat}:Hole mobility free/all- position\newline
x-axis:Position($nm$)\newline
y-axis:Mobility($m^{2} V^{-1} s^{-1}$)\newline
\newline
\textbf{nf.dat}:Free electron carrier density - position\newline
x-axis:Position($nm$)\newline
y-axis:Carrier density($m^{-3}$)\newline
\newline
\textbf{pf.dat}:Free hole carrier density - position\newline
x-axis:Position($nm$)\newline
y-axis:Carrier density($m^{-3}$)\newline
\newline
\paragraph{Model}
\textbf{imat.dat}:Material number - position\newline
x-axis:Position($nm$)\newline
y-axis:Number($au$)\newline
\newline
\paragraph{Recombination}
\textbf{Gn.dat}:Free electron generation rate - position\newline
x-axis:Position($nm$)\newline
y-axis:Generation rate($m^{-3} s^{-1}$)\newline
\newline
\textbf{Gp.dat}:Free hole generation rate - position\newline
x-axis:Position($nm$)\newline
y-axis:Generation rate($m^{-3} s^{-1}$)\newline
\newline
\textbf{Rn\_srh.dat}:SRH electron recombination rate - position\newline
x-axis:Position($nm$)\newline
y-axis:Recombination rate($m^{-3} s^{-1}$)\newline
\newline
\textbf{Rp\_srh.dat}:SRH hole recombination rate - position\newline
x-axis:Position($nm$)\newline
y-axis:Recombination rate($m^{-3} s^{-1}$)\newline
\newline
\textbf{R\_free.dat}:Free electron-hole recombination rate - position\newline
x-axis:Position($nm$)\newline
y-axis:Recombination rate($m^{-3} s^{-1}$)\newline
\newline
\textbf{fsrhh.dat}:Trap fermi level - position\newline
x-axis:Position($nm$)\newline
y-axis:Electron Fermi level($eV$)\newline
\newline
\textbf{fsrhn.dat}:Trap fermi level - position\newline
x-axis:Position($nm$)\newline
y-axis:Electron Fermi level($eV$)\newline
\newline
\textbf{nf\_to\_pt.dat}:Free electron to trapped hole - position\newline
x-axis:Position($nm$)\newline
y-axis:Rate($m^{-3} s^{-1}$)\newline
\newline
\textbf{nrelax.dat}:Electron relaxation rate - position\newline
x-axis:Position($nm$)\newline
y-axis:Rate($m^{-3} s^{-1}$)\newline
\newline
\textbf{pf\_to\_nt.dat}:Free hole to trapped electron - position\newline
x-axis:Position($nm$)\newline
y-axis:Rate($m^{-3} s^{-1}$)\newline
\newline
\textbf{prelax.dat}:Hole relaxation rate - position\newline
x-axis:Position($nm$)\newline
y-axis:Rate($m^{-3} s^{-1}$)\newline
\newline

\textbf{Photon\_gen.dat}:Photon generation rate - position\newline
x-axis:Position($nm$)\newline
y-axis:Photon generation rate($m^{-3} s^{-1}$)\newline

\paragraph{Transport}
\textbf{Jn.dat}:Current density - position\newline
x-axis:Position($nm$)\newline
y-axis:Electron current density($A m^{-2}$)\newline
\newline
\textbf{Jn\_diffusion.dat}:Diffusion current density - position\newline
x-axis:Position($nm$)\newline
y-axis:Electron current density (diffusion)($A m^{-2}$)\newline
\newline
\textbf{Jn\_drift.dat}:Drift current density - position\newline
x-axis:Position($nm$)\newline
y-axis:Electron current density (drift)($A m^{-2}$)\newline
\newline
\textbf{Jn\_plus\_Jp.dat}:Total current density (Jn+Jp) - position\newline
x-axis:Position($nm$)\newline
y-axis:Total current density (Jn+Jp)($A m^{-2}$)\newline
\newline
\textbf{Jp.dat}:Current density - position\newline
x-axis:Position($nm$)\newline
y-axis:Hole current density($A m^{-2}$)\newline
\newline
\textbf{Jp\_diffusion.dat}:Diffusion current density - position\newline
x-axis:Position($nm$)\newline
y-axis:Hole current density (diffusion)($A m^{-2}$)\newline
\newline
\textbf{Jp\_drift.dat}:Drift current density - position\newline
x-axis:Position($nm$)\newline
y-axis:Hole current density (drift)($A m^{-2}$)\newline
\newline
\textbf{Jp\_drift\_plus\_diffusion.dat}:Total current density (Jn+Jp) - position\newline
x-axis:Position($nm$)\newline
y-axis:Total current density (Jn+Jp)($A m^{-2}$)\newline
\newline

\newpage
\section{Data privacy statement}
In some versions of gpvdm it will ask you to register before using it.  In these versions it asks for your name, title, company that you work for and what you intend on using gpvdm for.  This data is then transmitted to the gpvdm server where it is securely stored. The reason I ask for this information is to be able generate accurate usage information. Having accurate information helps when requesting grants from funding bodies.  It's much easier to ask a funding body for money if you can prove you actually have users and your software is a benefit to society. Periodicity gpvdm will also contact the gpvdm server to see if there are any software updates. By using gpvdm you agree for the above to happen.


\newpage
\section{Copyright of the manual}
This manual is released under CC-BY license.


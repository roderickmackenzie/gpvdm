\newpage
\section{Automating/Scripting the model}
There are three main ways to automate the mode. The first is the parameter scan window, see section \ref{sec:scanwindow}.  This alows the user vary a paramter in steps. The second way is by using Python scripting, see section \ref{sec:pythonscripts} and the third way is through matlab scripting see section \ref{sec:matlabscripts}. 

\subsection{The parameter scan window}
\label{sec:scanwindow}

Sometimes one wishes to systematically vary a simulation parameter, to do this first bring up the parameter scan window, this can be done by clicking on the 




\begin{figure}[H]
\centering
\includegraphics[width=\textwidth]{./images/param_scan.png}
{\caption{Step 1: Select the 'Parameter scan' tool.}}
\label{overflow}
\end{figure}

Then make a new scan by clicking on the new button (1) then open the new scan by double clicking on the icon representing the scan (2). See figure \ref{fig:newscan}. This will bring up the scan window, see figure \ref{fig:newscanline}.

\begin{figure}[H]
\centering
\includegraphics[width=0.5\textwidth]{./images/param_scan_new.png}
\caption{Step 2: Make a new parameter scan, then double click on it to open it.}
\label{fig:newscan}
\end{figure}

\subsubsection{Changing one material parameter}

Once the scan window has appeared. Make a new scan line by clicking on the the plus icon \ref{fig:newscanline} (1), then select this line so that it is highlighted (2), then click on the three dots (3) to select which parameter you want to scan. In this example we will be selecting the electron mobility of a P3HT:PCBM solar cell. Do this by navigating to epitaxy$\rightarrow$ P3HT:PCBM$\rightarrow$ Drift diffusion$\rightarrow$ Electron mobility y. Highlight the parameter and then click OK. This should then appear in the scan line. The meaning of "epitaxy$\rightarrow$ P3HT:PCBM$\rightarrow$ Drift diffusion>Electron mobility y" will now be explained below:

\begin{itemize}
  \item epitaxy: All parameters in the .gpvdm file are exposed via the parameter selection window see \ref{fig:scanselect}. This file is a tree structure, see \ref{sec:gpvdmfileformat}. The all parameters which define the device it's self are contained under epitaxy.
  \item P3HT:PCBM: Under epitaxy each layer of the device is given by its name. The active layer in this device is called P3HT:PCBM.
  \item Drift diffusion: All electrical parameters are stored under drift diffusion.
  \item Electron mobility y: One can define asymmetric mobilities in the z,x and y direction - this is useful for OFET simulations.  However by default the model assumes a symmetric mobility which is the same in all directions. This value is defined by "Electron mobility y". 
\end{itemize}

\begin{figure}[H]
\centering
\includegraphics[width=\textwidth]{./images/param_scan_new_line.png}
\caption{Step 3: Add a 'scan line' to the scan.}
\label{fig:newscanline}
\end{figure}

\begin{figure}[H]
\centering
\includegraphics[width=0.5\textwidth]{./images/param_scan_select.png}
\caption{Step 5: Select the parameter you want to scan in the parameter selection window, in this case we are selecting epitaxy$\rightarrow$ P3HT:PCBM$\rightarrow$ Drift diffusion$\rightarrow$ Electron mobility y.}
\label{fig:scanselect}
\end{figure}

Next enter the values of mobility which you want to scan over in this case we will be entering "1e-5 1-6 1e-7 1e-8 1e-9" (see figure \ref{fig:runscan} 1) then click "run scan" (see figure \ref{fig:runscan} 2). Gpvdm will run one simulation on each core of your computer until all the simulations are finished.

\begin{figure}[H]
\centering
\includegraphics[width=\textwidth]{./images/param_scan_inputvalues.png}
\caption{Step 6: Enter the input values of mobility (or other values) you want to scan over (1). Then run the simulations.}
\label{fig:runscan}
\end{figure}

To view the simulation results click on the "output" tab this will bring up the simulation outputs, see figure \ref{fig:scanoutput}. You can see that a directory has been created for each variable that we scanned over so 1e-5, 1-6, 1e-7, 1e-8 and 1e-9.  If you look inside each directory it will be an exact copy of the base simulation directory.  If you double click on the files with multi colored JV curves, see the red box in figure \ref{fig:scanoutput}. Gpvdm will automaticity plot all the curves from each simulation in one graph, see figure \ref{fig:scanjv}.

\begin{figure}[H]
\centering
\includegraphics[width=\textwidth]{./images/param_scan_output.png}
\caption{Step 7: The output tab.}
\label{fig:scanoutput}
\end{figure}

\begin{figure}[H]
\centering
\includegraphics[width=0.5\textwidth]{./images/param_scan_jv.png}
\caption{Step 8: The result of the mobility scan.}
\label{fig:scanjv}
\end{figure}

\subsubsection{Duplicating parameters - changing the thickness of the active layer}

Very often one wants to change a parameter, then set another parameter equal to the parameter which was changed. An example of this is one may want to change electron and hole mobilities together when simulating a device with symmetric mobilities. This can be done using the duplicate function of the scan window as seen in figure \ref{fig:scanduplicate}.  In this example we tackle a slightly more trick problem than changing mobilities together we are going to change the physical width of the active layer and at the same time adjust the electrical mesh to make it match.  As discussed in section \ref{ref:mesh} the width of the active layer must always match the width of the electrical mesh.  When you change the layer width by hand in the layer editor gvpdm updates the width of the electrical mesh for you. But when scripting the model it won't do this update for you.  Therefore in the example below we are going to set the width of the active layer by scanning over:

epitaxy$\rightarrow$P3HT:PCBM$\rightarrow$dy of the object
\\
\\
Then we are going to add another line under and under parameter to scan select
\\
\\
mesh$\rightarrow$mesh\_y$\rightarrow$segment0$\rightarrow$len
\\
\\
and set it to
\\
\\
epitaxy$\rightarrow$P3HT:PCBM$\rightarrow$dy of the object
\\
\\
under the operation dropdown box. You will see the word duplicate appear under values.
\\
\\
If you now run the simulation "epitaxy$\rightarrow$P3HT:PCBM$\rightarrow$dy of the object" will be changed and "mesh$\rightarrow$mesh\_y$\rightarrow$segment0$\rightarrow$len" will follow it.


\begin{figure}[H]
\centering
\includegraphics[width=0.5\textwidth]{./images/param_scan_duplicate.png}
\caption{Duplicating material paramters.}
\label{fig:scanduplicate}
\end{figure}

\subsubsection{Setting constants}
Often when running a parameter scan one wants to set a constant value, this can be done using the "constant" option in the Operations dropdown menu. See figure \ref{fig:scanconst}

\begin{figure}[H]
\centering
\includegraphics[width=0.5\textwidth]{./images/param_scan_const.png}
\caption{The result of the mobility scan.}
\label{fig:scanconst}
\end{figure}



\subsubsection{Limitations of the scan window}
Although the scan window is convenient in that it provides a quick way to scan simulation parameters, it is by nature rather limited in terms of flexibility. If you want to do complex scans were multiple parameters are changed or to programmatically collect data from each simulation then you can use the python or matlab interfaces to gpvdm.  These are described in the next section.

\newpage
\subsection{Python scripting}
\label{sec:pythonscripts}

\newpage
\subsection{MATLAB scripting}
\label{sec:matlabscripts}


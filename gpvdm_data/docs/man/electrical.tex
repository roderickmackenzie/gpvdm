\newpage
\section{The drift diffusion electrical model}

\subsection{Outline}
Gpvdm's electrical model is a 1D/2D drift-diffusion model (like many others) however the special thing about gpvdm which makes it very good for disordered materials (Think organics, perovskites and a-Si) is that it goes to the trouble of explicitly solving the Shockley-Read-Hall equations as a function of energy and position space.  This enables one to model effects such as mobility/recombination rates changing as a function of carrier population and enables one to correctly model transients as one does not have to assume all the carriers in the trap states have reached equilibrium.  Things such as ToF transients, CELIV transients etc.. can be modelled with ease. Of course can be used for more ordered materials as well, you then just need to turn the traps off.

\subsection{Summary of model inputs}
A device is comprised of a series of layers (upto 10 layers), all these layers will interact with light.  Usually only one or two of these layers are electrically active, meaning the transport of electrons and holes must be modeled in detail.  Each electrically active layer with in the device has a set of electrical input parameters which define, charge transport, recombination and trapping. (see table)  If a device has more than one electrically active layer, then multiple sets of these parameters must be defined.  It should be noted that for organic materials (unlike inorganic) there is no standard set of material parameters for any given material.  The exact parameters will depend a lot on the fabrication conditions.  All layers in the device will also need a refractive index spectrum to be defined, this includes the real and imaginary refractive index as a function of wavelength (typically 300-1000 nm).
 

\subsection{Electrostatic potential}
The conduction band/valance band (or LUMO/HOMO in organic semiconductor speak) are defined as

\begin{equation}
E_{LUMO}=-\chi-q\phi
\end{equation}

\begin{equation}
E_{HOMO}=-\chi-E_g-q\phi
\end{equation}

To obtain the internal potential distribution within the device Poisson's equation is solved,

\begin{equation}
\label{eq:pos}
\nabla \cdot \epsilon_0 \epsilon_r \nabla = q (n_{f}+n_{t}-p_{f}-p_{t}-N_{ad}+-N_{ion}+a),
\end{equation}

where $n_{f}$, $n_{t}$ are the carrier densities of free and trapped electrons; $p_{f}$ and $p_{t}$ are the carrier densities of the free and trapped holes; and $N_{ad}$ is the doping density. $N_{ion}$ is the background density of perovskite ions and a is the density of mobile ions.

\subsection{Free charge carrier statistics}
For free carriers the model can either use Maxwell-Boltzmann statistics i.e.

\begin{equation}
n_{l}=N_c exp \left (\frac{F_n-E_{c}}{kT} \right)
\end{equation}

\begin{equation}
p_{l}=N_v exp \left(\frac{E_{v}-F_p}{kT} \right)
\end{equation}


or full Fermi-dirac statistics i.e.

\begin{equation}
n_{free}(E_{f},T)=\int^{\infty}_{E_{min}} \rho(E) f(E,E_{f},T) dE
\end{equation}

\begin{equation}
p_{free}(E_{f},T)=\int^{\infty}_{E_{min}} \rho(E) f(E,E_{f},T) dE
\end{equation}

where

\begin{equation}
f(E)=\frac{1}{1+e^{{E-E_f}/kT}}
\end{equation}

When using FD statistics free carriers are assumed to move in a parabolic band:

\begin{equation}
\rho(E)_{3D}=\frac{\sqrt{E}}{4\pi^2} \left ( \frac{2m^{*}}{\hbar^2}\right )^{3/2}
\end{equation}

The average energy of the carriers is defined as

\begin{equation}
\label{eq:energy}
\bar{W}(E_{f},T)=\frac{\int^{\infty}_{E_{min}} E \rho(E) f(E,E_{f},T) dE}{\int^{\infty}_{E_{min}} \rho(E) f(E,E_{f},T) dE}
\end{equation}

\subsection{Carrier trapping and Shockley-Read-Hall recombination}

The model provides two methods to account for carrier trapping and recombination via trap states.  The first by equation \ref{eq:ss_srh}, this assumes that the trapped carrier distribution has reached equilibrium.  It also assumes there are relatively few trapped charge carriers compared the the number of free carriers, and thus the trapped charges do not significantly change the electrostatic potential.  These assumptions are valid when the material is very ordered (i.e. GaAs) or at a push in steady state for some moderately disordered material systems. However if you wish to simulate transient or frequency domain experiments, then you can no longer use \ref{eq:ss_srh}.  Instead, one must use a non-equilibrium SRH approach which does not assume trapped carriers have reached equilibrium.  Unlike many other models, gpvdm has such a non-equilibrium SRH model built in this is described in section \ref{sssec:dynamic}. In fact, it is turned on by default so when using gpvdm you have to go out of your way to turn on equation \ref{eq:ss_srh}.

To understand the importance of such a dynamic solver, consider the following example: You are performing a transient photocurrent experiment (TPC). You photo-excite your device with a laser, carriers very quickly become trapped during the first 1-2$\mu s$ after photoexcitation, as time passes, the carriers gradually de-trap from deeper and deeper trap states and produce the long photocurrent transient \cite{mackenzie2013interpreting}. These transients can often extend out to over 1 second after photo-excitation.  Current at the start of the transient originates from shallow traps while current at the end of the transient originates from carriers from very deep trap levels. To simulate this one has to be able to account for the gradual emptying of trap states firstly starting at the shallow traps, then progressing to deeper and deeper trap states. Were one to assume all trap states were in equilibrium one would not be able to simulate this process.

So in summary, although many others have used \ref{eq:ss_srh} to model disordered devices in time DON'T you results won't make sense. If you want to simulate anything but steady state in an ordered device turn ON the non-equilibrium solver.

\subsubsection{Equilibrium Shockley-Read-Hall recombination}

For some very ordered material systems where there are not many trap states it is enough to describe SRH trap states using the equation:

\begin{equation}
\label{eq:ss_srh}
R^{SRH}=\frac{np-n_{0}*p_{0}}{\tau_{p} (n+n_{1})+\tau_{n} (p+p_{1})}
\end{equation}

%/https://www.iue.tuwien.ac.at/phd/ayalew/node72.html
 where $R_{SRH}$ is the rate of SRH recombination, $n,p$ are the density of free charge carriers $n_0, p_0$, are the equilibrium density of charge carriers, $\tau_{n,p}$ are the SRH life times and $n_{1}$ and $p_{1}$ are the trapped electron and hole densities when the Fermi-level matches the trap state energy.  This can be turned on in the electrical parameter editor.

\subsection{Carrier trapping and Shockley-Read-Hall recombination}


\begin{figure}
To describe charge becoming trapping into trap states and recombination associated with those states the model uses Shockley-Read-Hall (SRH) theory. A 0D depiction of this SRH recombination and trapping is shown in figure \ref{fig:dos_structure}, the free electron and hole carrier distributions are labeled as n free and p free respectively. The trapped carrier populations are denoted with n trap and p trap , they are depicted with filled red and blue boxes. SRH theory describes the rates at which electrons and holes become captured and escape from the carrier traps. If one considers a single electron trap, the change in population of this trap can be described by four carrier capture and escape rates as depicted in figure \ref{fig:dos_structure}. The rate rec describes the rate at which electrons become captured into the electron trap, $r_{ee}$ is the rate which electrons can escape from the trap back to the free electron population, $r_{hc}$ is the rate at which free holes get trapped and $r_{he}$ is the rate at which holes escape back to the free hole population. Recombination is described by holes becoming captured into electron space slice through our 1D traps. Analogous processes are also defined for the hole traps.

\centering
\includegraphics[width=40mm]{./images/dos_structure.jpg}
\caption{Trap filling in both energy and position space as the solar cell is taken from a negative bias
Carrier trapping, de-trapping, and recombination}
\label{fig:dos_structure}
\end{figure}


\begin{table}
\begin{center}
  \begin{tabular}{lll}
  \hline
  Mechanism & Symbol & Description  \\
  \hline
Electron capture rate & $r_{ec}$ & $n v_{th} \sigma_{n} N_{t}(1-f)$ \\
Electron escape rate & $r_{ee}$ & $e_{n} N_{t} f$ \\
Hole capture rate & $r_{hc}$ & $p v_{th} \sigma_{p} N_{t} f$ \\
Hole escape rate & $r_{he}$ & $e_{p} N_{t} (1-f)$\\
  \hline
\end{tabular}
\end{center}
\caption{Shockley-Read-Hall trap capture and emission rates, where $f$ is the fermi-Dirac occupation function and $N_{t}$ is the trap density of a single carrier trap.}
\label{tab:rates}
\end{table}


For each trap level the carrier balance \ref{eq:srhrate} is solved, giving each trap level an independent quasi-Fermi level. Each point in position space can be allocated between 10 and 160 independent trap states.  The rates of each process $r_{ec}$, $r_{ee}$, $r_{hc}$, and $r_{he}$ are give in table \ref{tab:rates}.

\begin{equation}
\label{eq:srhrate}
\frac{\delta n_t}{\partial t}=r_{ec}-r_{ee}-r_{hc}+r_{he}
\end{equation}

The escape probabilities are given by:

\begin{equation}
\label{eq:taile}
e_n=v_{th}\sigma_{n} N_{c} exp \left ( \frac{E_t-E_c}{kT}\right )
\end{equation}

and

\begin{equation}
\label{eq:taile}
e_p=v_{th}\sigma_{p} N_{v} exp \left ( \frac{E_v-E_t}{kT}\right )
\end{equation}

 where $\sigma_{n,p}$ are the trap cross sections, $v_{th}$ is the thermal emission velocity of the carriers, and $N_{c,v}$ are the effective density of states for free electrons or holes.  The distribution of trapped states (DoS) is defined between the mobility edges as

\begin{equation}
\label{eq:taile}
\rho^{e/h}(E)=N^{e/h}exp(E/E_{u}^{e/h})
\end{equation}

where , $N_{e/h}$ is the density of trap states at the LUMO or HOMO band edge
in states/eV and where $E_{U}^{e/h}$ is slope energy of the density of states. 

The value of $N_{t}$ for any given trap level is calculated by averaging the DoS function over the energy ($\Delta E$ ) which a trap occupies:

\begin{equation}
\label{eq:taile}
N_{t}(E)=\frac{\int^{E+\Delta E/2}_{E-\Delta E/2} \rho^{e}{E} dE}{\Delta E}
\end{equation}

The occupation function is given by the equation,
\begin{equation}
f(E_{t},F_{t})=\frac{1}{e^{\frac{E_{t}-F_{t}}{kT}}+1}
\end{equation}
Where, $E_{t}$ is the trap level, and $F_{t}$ is the Fermi-Level of the trap.
The carrier escape rates for electrons and holes are given by



\subsubsection{Free-to-free carrier recombination}
A free-carrier-to-free-carrier recombination pathway is also included in the model. However, most organic solar cells have a great deal of trap states and an ideality factor greater than 1.0 suggesting that free to free recombination is not the dominant mechanism.  It is also worth noting that since I included this recombination pathway in the model I have not found it useful to reproduce experimental results.

Free-to-free recombination is described using equation \ref{equ:freetofree}

\begin{equation}
R_{free}=k_{r}(n_{f}p_{f}-n_{0}p_{0})
\label{equ:freetofree}
\end{equation}



\subsubsection{Free-to-free carrier recombination}
A free-carrier-to-free-carrier recombination (bi-molecular) pathway is also included. However, most organic solar cells have a great deal of trap states and an ideality factor greater than 1.0 suggesting that free-to-free recombination is not the dominant mechanism.  Free-to-free recombination is described using equation \ref{equ:freetofree}

\begin{equation}
R_{free}=k_{r}(n_{f}p_{f}-n_{0}p_{0})
\label{equ:freetofree}
\end{equation}

\subsubsection{Auger recombination}

Auger recombination is as

\begin{equation}
R^{AU}=(C^{AU}_{n}n+C^{AU}_{p}p)(np-n_{0}p_{0})
\end{equation}

where $C^{AU}_{n}$ and $C^{AU}_{p}$ are the Auger coefficient of electrons and holes in $m^6 s^{-1}$. This can be set in the electrical paramter editor.

%https://www.iue.tuwien.ac.at/phd/ayalew/node73.html


\subsubsection{Geminate recombination - organics only}
There are a number of models to calculate the number of geminate pairs which get converted to free charge carriers (i.e. the Onsager-Braun model).  However, these models will generally require a number of parameters which are not reliably known. So although it's possible (and interesting) to write a model to simulate geminate recombination, one is usually better off simply introducing a photon efficiency factor by which the number of absorbed photons is multiplied. This factor can be obtained to a reasonable degree by comparing the difference between the simulated and experimental $J_{sc}$.  This parameter can set in the configuration section of the optical simulation window.


\subsection{Charge carrier transport}
To describe charge carrier transport, the bi-polar drift-diffusion equations are solved in position space
for electrons,
\begin{equation}
\label{eq:ndrive}
\boldsymbol{J_n} = q \mu_e n_{f}  {\nabla E_{c}}  + q D_n  {\nabla n_{f}},
\end{equation}
and holes,
\begin{equation}
\label{eq:pdrive}
\boldsymbol{J_p} = q \mu_h p_{f}  {\nabla E_{v}}  - q D_p {\nabla p_{f}}.
\end{equation}

Conservation of charge carriers is forced by solving the charge carrier continuity equations for both electrons,
\begin{equation}
\label{eq:contn}
\nabla \boldsymbol{J_n}  = q (R-G+\frac{\partial n}{\partial t}),
\end{equation}
and holes
\begin{equation}
\label{eq:contp}
\nabla \boldsymbol{J_p} = - q (R-G+\frac{\partial p}{\partial t}).
\end{equation}

where $R$ and $G$ are the net recombination and generation rates per unit volume respectively.

\section{Perovskite mobile ion solver}
The mobile ion solver is implemented after the work of Calado \cite{calado2016evidence}

\begin{equation}
\label{eq:pdrive}
\boldsymbol{J_a} = q \mu_a a_{f}  {\nabla E_{v}}  - q D_a {\nabla a_{f}}.
\end{equation}

\begin{equation}
\label{eq:contp}
\nabla \boldsymbol{J_a} = - q \frac{\partial a}{\partial t}.
\end{equation}



\subsection{Semiconductor interfaces}
\subsubsection{Tunnelling through heterojunctions}
Tunnelling of holes through hetrojunction interfaces are is give by
\begin{equation}
\boldsymbol{J_p} = q T_{h}  ((p_{1}-p_{1}^{eq})-(p_{0}-p_{0}^{eq})),
\end{equation}

and for electrons

\begin{equation}
\boldsymbol{J_n} = -q T_{e}  ((n_{1}-n_{1}^{eq})-(n_{0}-n_{0}^{eq})).
\end{equation}

Where $T_{h}$ and $T_{e}$ represent the rate constants of the tunnelling. This can be configured in the interfaces editor.

\subsubsection{Doping on the interface}
Using the interface editor, layers of doping measuring one mesh point thick can be added to either side of the interface.  This is useful for OFET simulations where interface charge is important to the turn on voltage.


\subsection{Calculating the built in potential}  \label{sssec:initial}
The first step to performing a device simulation, is to calculate the built in potential of the device.  To do this we must know the following things:

\begin{itemize}

  \item The majority carrier concentrations on the contacts $n$ and $p$.
  \item The effective densities of states $N_{LUMO}$ and $N_{HOMO}$.
  \item The effective band gap $E_g$

\end{itemize}

\begin{figure}[H]
\centering
\includegraphics[width=120mm]{./images/bands.png}
\caption{Band structure of device in equilibrium.}
\label{fig:bands}
\end{figure}

\vspace{1em}
The left hand side of the device is given a reference potential of 0 V.  See figure \ref{fig:bands}.  We can then write the energy of the LUMO and HOMO on the left hand side of the device as:

\begin{equation}
E_{LUMO}=-\chi
\end{equation}

\begin{equation}
E_{HOMO}=-\chi-E_{g}
\end{equation}

For the left hand side of the device, we can use Maxwell-Boltzmann statistics to calculate the equilibrium Fermi-level ($F_i$).

\begin{equation}
p_{l}=N_v exp \left(\frac{E_{HOMO}-F_p}{kT} \right)
\end{equation}

We can then calculate the minority carrier concentration on the left hand side using $F_i$

\begin{equation}
n_{l}=N_c exp \left (\frac{F_n-E_{LUMO}}{kT} \right)
\end{equation}

The Fermi-level must be flat across the entire device because it is in equilibrium.  However we know there is a built in potential, we can therefore write the potential of the conduction and valance band on the right hand side of the device in terms of $phi$ to take account of the built in potential.

\begin{equation}
E_{LUMO}=-\chi-q\phi
\label{equ:Ev_rhs}
\end{equation}

\begin{equation}
E_{HOMO}=-\chi-E_g-q\phi
\end{equation}

we can now calculate the potential using

\begin{equation}
n_{r}=N_c exp \left (\frac{F_n-E_{LUMO}}{kT} \right)
\end{equation}
equation \ref{equ:Ev_rhs}.

The minority concentration on the right hand side can now also be calculated using.

\begin{equation}
p_{r}=N_v exp \left (\frac{E_v-F_{HOMO}}{kT} \right)
\end{equation}

The result of this calculation is that we now know the built in potential and minority carrier concentrations on both sides of the device.  Note, infinite recombination velocity on the contacts is assumed.  I have not included finite recombination velocities in the model simply because they would add four more fitting parameters and in my experience I have never needed to use them to fit any experimental data I have come across.

Once this calculation has been performed, we can estimate the potential profile between the left and right hand side of the device, using a linear approximation. From this the charge carrier densities across the device can be guessed.  The guess for potential and carrier densities, is then used to prime the main Newton solver.  Where the real value are calculated.  The Newton solver is described in the next section.



\subsubsection{Average free carrier mobility}
In this model there are two types of electrons (holes), free electrons (holes) and trapped electrons (holes).  Free electrons (holes) have a finite mobility of $\mu_e^0$ ($\mu_h^0$) and trapped electrons (holes) can not move at all and have a mobility of zero.  To calculate the average mobility we take the ratio of free to trapped carriers and multiply it by the free carrier mobility.:

\begin{equation}
\mu_e(n)=\frac{\mu_e^0 n_{free}}{n_{free}+n_{trap}}
\end{equation}

Thus if all carriers were free, the average mobility would be $\mu_e^0$ and if all carriers were trapped the average mobility would be 0.  It should be noted that only $\mu_e^0$ ($\mu_h^0$) are used in the model for computation and $\mu_e(n)$ is an output parameter.

The value of $\mu_e^0$ ($\mu_h^0$) is an input parameter to the model.  This can be edited in the electrical parameter editor.  The value of $\mu_e(n)$, and $\mu_h(p)$ are output parameters from the model.  The value of $\mu_e(n)$, and $\mu_h(p)$ change as a function of position, within the device, as the number of both free and trapped charge carriers change as a function of position.  The values of  $\mu_e(x)$, and $\mu_h(x)$ can be found in $mu\_n\_ft.dat$ and $mu\_p\_ft.dat$ within the $snapshots$ directory.  The spatially averaged value of mobility, as a function of time or voltage can be found in the files $dynamic\_mue.dat$ or $dynamic\_muh.dat$ within the dynamic directory.

Were one to try to measure mobility using a technique such as CELIV or ToF, one would expect to get a value closer to $\mu_e(n)$ or $\mu_h(p)$ rather than closer to $\mu_e^0$ or $\mu_h^0$.  It should be noted however, that measuring mobility in disordered materials is a difficult thing to do, and one will get a different experimental value of mobility depending upon which experimental measurement method one uses, furthermore, mobility will change depending upon the charge density profile within the device, and thus upon the applied voltage and light intensity.  To better understand this, try for example doing a CELIV simulation, and plotting $\mu_e(n)$ as a function of time (Voltage).  You will see that mobility reduces as the negative voltage ramp is applied, this is because carriers are being sucked out of the device.  Then try extracting the mobility from the transient using the CELIV equation for extracting mobility.  Firstly, the CELIV equation will give you one value of mobility, which is a simplification of reality as the value really changes during the application of the voltage ramp.  Secondly, the value you get from the equation will almost certainly not match either $\mu_e^0$ or any value of $\mu_e(n)$.  This simply highlights, the difficult of measuring $a$ value of mobility for a disordered semiconductor and that really when we quote a value of mobility for a disordered material, it really only makes sense to quote a value measured under the conditions a material will be used.  For example, for a solar cell, values of $\mu_e(n)$ and $\mu_h(n)$, would be most useful to know under 1 Sun at the $P_{max}$ point on a JV curve.


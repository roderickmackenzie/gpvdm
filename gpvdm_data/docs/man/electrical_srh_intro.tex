\section{Carrier trapping and Shockley-Read-Hall recombination}
\label{sec:SRHintro}
The model provides two methods to account for carrier trapping and recombination via trap states.  The first by equation \ref{eq:ss_srh}, this assumes that the trapped carrier distribution has reached equilibrium.  It also assumes there are relatively few trapped charge carriers compared the the number of free carriers, and thus the trapped charges do not significantly change the electrostatic potential.  These assumptions are valid when the material is very ordered (i.e. GaAs) or at a push in steady state for some moderately disordered material systems. However if you wish to simulate transient or frequency domain experiments, then you can no longer use \ref{eq:ss_srh}.  Instead, one must use a non-equilibrium SRH approach which does not assume trapped carriers have reached equilibrium.  Unlike many other models, gpvdm has such a non-equilibrium SRH model built in this is described in section \ref{sssec:dynamic}. In fact, it is turned on by default so when using gpvdm you have to go out of your way to turn on equation \ref{eq:ss_srh}.

To understand the importance of such a dynamic solver, consider the following example: You are performing a transient photocurrent experiment (TPC). You photo-excite your device with a laser, carriers very quickly become trapped during the first 1-2$\mu s$ after photoexcitation, as time passes, the carriers gradually de-trap from deeper and deeper trap states and produce the long photocurrent transient \cite{mackenzie2013interpreting}. These transients can often extend out to over 1 second after photo-excitation.  Current at the start of the transient originates from shallow traps while current at the end of the transient originates from carriers from very deep trap levels. To simulate this one has to be able to account for the gradual emptying of trap states firstly starting at the shallow traps, then progressing to deeper and deeper trap states. Were one to assume all trap states were in equilibrium one would not be able to simulate this process.

So in summary, although many others have used \ref{eq:ss_srh} to model disordered devices in time DON'T you results won't make sense. If you want to simulate anything but steady state in an ordered device turn ON the non-equilibrium solver.
